% !TeX encoding = UTF-8

% What libraries, existing algorithms were used?

\section{Methods}

The input to the detection algorithm is an RGB image of a Mondrian painting. The
expected output is a list of rectangles with their position, sizes and color as
they are conceptionally seen in the painting. All rectangles combined are
expected to exactly cover the area of the input image in a way that the original
could be dissected into this set of rectangles.

The algorithm itself can be separated into two different phases: Image
preprocessing and the recognition of rectangles.

The goal of the first preprocessing phase is to get a binary image (an image
with only black and white pixels) in which the black represent the lines of the
painting while the white represents the inner area of the rectangles.

The second phase then takes this binary image and returns a list of found
rectangles. Additionally, for recognizing the colors of the rectangles the
original input image is used.

\subsection{Image preprocessing}

At first, the input image is transformed with a Gaussian blur to reduce
artifacts that the input images might include.

Next, the RGB input image is decomposed into a greyscale image by always using
the maximum value of the RGB triplet for each pixel.

To normalize the greyscale image and increase the contrast of the darker
regions, we now apply histogram equalization. Since global equalization did not
work well for all images, we instead used Contrast Limited Adaptive Histogram
Equalization (CLAHE) from OpenCV. In contrast to ordinary adaptive histogram
equalization algorithms, this prevents the overamplification of noise. % TODO:
Better explain Global harmonization -> Adaptive harmonization -> CLAHE

After the normalization is applied, we now further brighten colorful areas of
the image to make them further stand out against the black areas of the
paintings. This way, for example, darker blue areas can be better distinguished
from black areas. Since the difference between the maximum and the minimum value
of the RGB triplet is related to the colorfulness of a color, we calculate it
for every pixel of the input image. The resulting mask from this calculation is
added to the contrast-normalized image.

Now the image is converted into a binary image using an arbitrary threshold for
the greyscale values. This value of the threshold can be configured. Through
test runs a value of 110 for a greyscale range of 0 to 255 showed to give the
reasonable results.

Since the thresholding step might also turn a few areas white that a human would
recognize as part of a black line an erosion is applied now. Erosion is an image
morphology that increases the size or black areas in an image using a binary
matrix called the \textit{structuring element}. The center of this matrix is
aligned with each input pixel with the value 0 (black). Then the shape of the
matrix is removed from the input image. In our case, the structuring element is
a $N\times N$ filled matrix. Using the erosion possible considered accidental
interruptions in the lines can be restored. However choosing a structuring
element that is too big, might result into loss of information. Therefore the
size $N$ of the erosion kernel needs to be chosen carefully.

The resulting image now separates the darker parts of the image fairly well.
However, we are only interested in black lines but Mondrian paintings also
include filled black rectangles. To remove these black rectangles we create a
mask that applies a dilation with a comparatively large structuring element on
the image. A dilation is the opposite of an erosion, increasing the size of
white areas. The size of the structuring element is chosen in a way that most of
the lines in the paintings are removed, ideally only leaving inner subsections
of black rectangles. The resulting mask is then removed from the image of the
last step. Hence only the outlines of the black rectangles remain.

\subsection{Detection and recognition of rectangles}

To detect the different rectangles in the resulting binary image is scanned
horizontally as well as vertically line by line for each pixel to find
horizontal and vertical lines. It only recognizes lines with a certain minimum
length. However, this means that what we perceive as for example a horizontal
line in the painting is now recognized as multiple lines next to each other. A
line that is 50 pixels wide would be recognized as 50 lines.

To then actually find the perceived lines, parallel lines are merged into one
line. As long as parallel lines are within a certain distance from each other
they are merged into one line. For horizontal lines, the resulting line will
have the average y value of all those lines. And the minimum and maximum x
values for starting and end points.

From out idealistic view on Mondrian paintings, the ends of lines always touch
another line or the edge of the painting. Therefore the next step of the
algorithm is to look at the ends of each line, determine the closest line and
change the end of the line to this closest line. To simplify it the edges of the
paintings are added as lines.

These lines now depict a number of rectangles. To find these rectangles we first
find all the corners of the rectangles. The corners are always a cross-section
of two lines. We find the top-left, top-right, bottom-left and bottom-right
corners.

After this four different corners are combined into a rectangle by finding
matching corners.

Now that we have a list of rectangles, we determine the colors of the
rectangles. For this purpose, we clip the rectangle from the original image and
determine the average color. This color is then compared to black, white, red,
blue and yellow. The colors that we expect in an idealistic Mondrian painting.
