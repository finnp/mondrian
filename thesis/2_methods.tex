% !TeX encoding = UTF-8
\section{Methods}

% How does the program work?
% What libraries, existing algorithms were used?
% Description of the automatic Image processing
% Detection of shapes

Algorithm for detecting rectangles in Mondrian paiting can be separated roughly
in two parts:
\begin{itemize}
  \item Image preprocessing
  \item Detection and recognition of rectangles
\end{itemize}

\subsection{Image preprocessing}
\begin{itemize}
  \item Gaussian blur
  \item Turning the image into a greyscale (Finding the "brightest" part of the RGB, black and white)
  \item Adjusting the contrast of the image using Contrast Limited Adaptive Histogram Equalization
  \item Further brighten colorful areas to better separate them from black areas
  \item Converting the image into a binary image via a binary threshold
  \item Using the image morphology erosion to increase the size of black areas
  \item Turn white rectangles in white ones with black borders (via a mask and a dilation)
\end{itemize}

\subsection{Detection and recognition of rectangles}
\begin{itemize}
  \item Find all horizontal and vertical lines by going through all the pixels
  \item Combine close parallel lines to one line
  \item Connect the lines so that the ends of each line always touch another line
  \item Find the four different kind of corners that these lines form
  \item Find the different rectangles given these corners
  \item Determine the color of the rectangles by comparing the area with the original
\end{itemize}}

% Siehe \zb \cite{Dje06, DjeOezSal07, CocWil00}.
