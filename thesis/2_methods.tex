% !TeX encoding = UTF-8

\section{Fundamentals}

For preprocessing the images, a combination of methods are used: \textit{Image
Morphology}, \textit{Thresholding}, \textit{Contrast Limited Adaptive Histogram
Equalization (CLAHE)} and \textit{Masks}. These algorithms are provided by the
Python distribution of the Open Source library OpenCV \cite{opencv_library}.

\subsection{Definitions}

We define an Image $I$ with width $w$ and height $h$ to be the set
$I = \{ p_{xy} | x \in 1 \dots w, y \in 1 \dots h \}$. A RGB Image $I$ is an
Image with $p_{xy} \in (r_{xy}, g_{xy}, b_{xy})$, while a Greyscale image is
an Image with $p_{xy} \in 0,1, \dots 255$.

A Binary Image $B$ is an Image with $p_{xy} \in \{0,1\}$ where we also refer to
$0$ as \textit{black} and $1$ as \textit{white}

\subsection{Image Morphology}

Image Morphology is based on Mathematical Morphology from the mathematical field
of set theory. We are going to present two basic morphological operators
commonly used in image processing without going into detail of their mathematical
definition. % TODO: Further reading in ...

Given a binary image $X$ and a smaller binary image $B$ called the
\textit{Structuring Element} with an anchor point $a \in B$,
we can apply the following operations.

The Erosion $X \ominus B$ is equivalent to iterating through all of the white
pixels of $X$ and only keeping them if all of the points of $B$ are translated
by its anchor point $a$ to this position is fully included in $X$. Therefore the
white areas in $X$ are reduced \cite{Smith1997}.

%TODO: Explain the Dilation section in more detail

Similarly with the Dilation operation $X \oplus B$ neighbouring pixels are
turned white if they intersect with the translated Structuring Element. The size
of white areas is increased.

Image Morphologies have multiple uses like increasing certain areas for better
recognition or separating larger shapes of an image.

\subsection{Concept}

% TODO: Add introductory sentence to his setion (Aim for detection Alg...)

% TODO: Explain what happened to the lines

The input to the detection algorithm is an RGB Image $I_1$ of a Mondrian
painting. The expected output is a list of rectangles with their position, sizes
and color as they are conceptionally seen in the painting. All rectangles
combined are expected to exactly cover the area of the input image in a way that
the original could be dissected into this set of rectangles.

The algorithm itself can be separated into two different phases: Image
preprocessing (\ref{preprocessing}) and the recognition of rectangles (\ref{rectangles}).

The goal of the first preprocessing phase is to obtain a binary image (an image
with only black and white pixels) in which black represent the lines of the
painting while white represents the inner area of the rectangles.

The second phase then takes this binary image and returns a list of found
rectangles. Additionally, for recognizing the colors of the rectangles, the
original input image is used.

\subsubsection{Image preprocessing} \label{preprocessing}

% TODO: Explain where artifacts might come from

At first, a Gaussian blur is applied to the input RGB Image
$I_1$ to reduce artifacts that the input images might include: $I_2 = G(I_1)$

Next, the RGB input image is decomposed into a Greyscale Image $G_1$ by using
the maximum value of the RGB triplet for each pixel in $I_2$.

% TODO: Exaplain and add source why CLAHE is better than standard equalisation

To normalize the greyscale image and increase the contrast of the darker
regions, we now apply histogram equalization to $G_1$. Since global equalization
did not work well for all images, we instead used \textit{Contrast Limited
Adaptive Histogram Equalization (CLAHE)} from OpenCV. In contrast to
conventional adaptive histogram equalization algorithms, this prevents the
overamplification of noise. We call the result of the equalization $G_2$.

After the normalization is applied, we further brighten colorful areas of the
image to make them stand out against the black areas of the paintings. This way,
for example, darker blue areas can be better distinguished from black areas.
Since the difference between the maximum and the minimum value of the RGB
triplet is related to the colorfulness of a color, we calculate it for every
pixel of the input image. The resulting Mask $G_m$ from this calculation is added to
the contrast-normalized image $G_3 = G_2 + G_m$.

Now the image $G_3$ is converted into a binary image $B_1$ using an arbitrary
threshold for the greyscale values. This value of the threshold $t$ can be
configured.

% TODO: How is N choosen? Why can it result in loss of information?

Since the thresholding step might leave some interruption in the black lines, we
reduce the white areas of the image: An Erosion $B_2 = B_1 \ominus B_e$ is
applied. In this case, the structuring element $B_e$ is a $N\times N$ binary
Image. Using the Erosion some accidental interruptions in the lines can be
restored. However choosing a the Structuring Element that is too high, might
result in loss of information. Therefore the size $N$ needs to be chosen
carefully.

The resulting image $B_2$ now separates the darker parts of the image fairly
well. However, we are only interested in black lines, but Mondrian paintings
also include filled black rectangles. To remove these black rectangles, we
create a mask that applies a Dilation $B_m = B_2 \oplus B_d$ with a larger
Structuring Element $B_d$ compared with $B_e$ on the image. The size of $B_d$ is
chosen in a way that most of the lines in the paintings are removed, only
leaving inner areas of black rectangles. The resulting mask $B_m$ is then
removed from the image of the last step $B_3 = B_2 - B_m$. Hence only the
outlines of the black rectangles, as well as the black lines, remain in $B_3$.

%TODO: Create an image showing the preprocessing steps

\subsubsection{Detection and recognition of rectangles} \label{rectangles}

Since the rectangles in the input image are defined by the horizontal and
vertical lines in the image, the detection starts by finding all of those lines
in the output binary image from the previous phase.

%TODO: Explain Horizontal vs. Vertical better

To find horizontal lines, the image is scanned line by line for each pixel.
Uninterrupted sequences of black pixels are added to a list of horizontal lines.
Only sequences with a specified minimum length are added. The minimum length
should ideally be slightly larger than the maximum width of the lines in the
image. Vertical lines are recognized respectively. This step is the equivalent of
applying a lossy run-length encoding once horizontally and once vertically.

However, this means that what we perceive as one horizontal line in the painting
is now recognized as multiple lines next to each other. For example, a line that
is 50 pixels wide would be recognized as 50 lines. Therefore as a next step
parallel lines close to each other are merged into a single line. As long as
parallel lines are within a certain distance from each other, they are merged
into one line. For horizontal lines, the resulting line will have the average
$y$ value of all those lines and the minimum and maximum $x$ values as starting
and end points.

From out idealistic view on Mondrian paintings, the ends of lines always touch
another line or the edge of the painting. However, the lines that we reduced now
might overlap slightly or not even connect to the next line.

% TODO: Specify how to find the closest line

We, therefore, iterate over all the ends of the lines, determine the closest
line to this point and change the position of this end to the closest line. For
the purpose of this algorithm, the edges of the image are considered to be lines
as well.

%TODO: How are they identified as top-left?

All lines should now represent the structure of the painting. What we want as
our output though is a list of rectangles. Every rectangle in the image can be
defined through a set of four different corners: top-left, top-right,
bottom-left and bottom-right. These corners are always intersections of two
lines, either crossing or touching. We determine the corners and their types by
iterating through the lines.

After this, four different corners are combined into a rectangle by finding
matching corners. This is done by iterating through the top-left corners $(x,y)$
and finding the closest top-right corner to the right $(x_r, y)$ and the
closest bottom-left corner below $(x, y_b)$. The rectangle is then defined by
the position of the top-left corner and its width and height
$(x,y,x_r-x,y_b-y)$.

For this list of rectangles, we determine the colors of the rectangles. For this
purpose, we clip the rectangle from the original image and determine the average
color or the selection. This color is then reduced to either black, white, red,
blue or yellow.

\section{Implementation}

%TODO: Write implementation

Challenges?
Code: Pseudo/Python?
How do the choose threshold t?
Min length t?
Distance d?
How to choose N?

Thresholding:
Through test runs a value of 110 for a greyscale range of 0 to 255
showed to give the reasonable results.
