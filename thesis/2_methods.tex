% !TeX encoding = UTF-8
\section{Methods}

% What libraries, existing algorithms were used?

Algorithm for detecting rectangles in Mondrian paiting can be separated roughly
in two parts:
\begin{itemize}
  \item Image preprocessing
  \item Detection and recognition of rectangles
\end{itemize}

\subsection{Image preprocessing}

The goal of the image preprocessing step is to turn the input image into a
black-and-white or binary image where the lines of a Mondrian paiting are black
while the inner side of the rectangles are white.

At first the image is transformed with a minimal Gaussian blur to reduce
artifacts that the input images might include. (Reference Gaussian blur paper?)

After that the RGB input image is converted into a greyscale image by using the
maximum value of the RGB triplet.

Now the contrast of the image is harmonized using the Contrast Limited Adaptive
Histogram Equalization algorithm.

As the next step the colorful areas of the images are further brightened to make
them further stand out against the black areas of the paintings. This way darker
blue areas can be better distinguished compared to black areas.

Now the image is converted into a binary image using an arbitrary threshold that
can be configured.

Since this step will likely also turn a few areas white that a human would
recognize as black line an erosion is applied now. Erosion is an image morphology
that increases the size or black areas in an image. This way possible unwanted
cracks in the lines are restored.

The resulting image now separes the darker parts of the image fairly well.
However we are only interested into black lines, but Mondrian paitings also include
black rectangles. To remove these black rectangles we create a mask that applies a
comparatively strong dilation on the image (the oppotive of an erosion). The resulting
mask is then removed from the image in the step before. This way only the outline of the
black rectangles remains. (This could actually be shown as a formular, A - Erode(A))


\subsection{Detection and recognition of rectangles}

To detect the different rectangles in the resulting binary image is scanned
horizontally as well as vertically line by line for each pixel to find horizontal
and vertical lines. It only recognizes lines with a certain minimum length.
However this means that what we perceive as for example a horizontal line in the
painting is now recognized as multiple lines next to each other. A line that is
50 pixel wide would be recognized as 50 lines.

To then actually find the perceived lines, parallel lines are merged into one line.
As long as parallel lines are within a certain distance from each other they are
merged into one line. For horizontal lines the resulting line will have the average
y value of all those lines. And the minimum and maximum x values for starting and end
points.

From out idealistic view on Mondrian paintings, the ends of lines always touch
another line or the edge of the painting. Therefore the next step of the algorithm
is to look at the ends of each line, determine the closest line and change the
end of the line to this closest line. To simplify it the edges of the paintings
are added as lines.

These lines now depict a number of rectangles. To find these rectangles we
first find all the corners of the rectangles. The corners are always a cross-section
of two lines. We find the top-left, top-right, bottom-left and bottom-right corners.

After this four different corners are combined into a rectangle by finding matching
corners.

Now that we have a list of rectangles, we determine the colors of the rectangles.
For this purpose we clip the rectangle from the original image and determine the
average color. This color is then compared to black, white, red, blue and yellow.
The colors that we expect in an idealistic mondrian painting.
