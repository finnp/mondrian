% !TeX encoding = UTF-8
\section{Methods}

% How does the program work?
% What libraries, existing algorithms were used?
% Description of the automatic Image processing
% Detection of shapes

Algorithm for detecting rectangles in Mondrian paiting can be separated roughly
in two parts:
\begin{itemize}
  \item Image preprocessing
  \item Detection and recognition of rectangles
\end{itemize}

\subsection{Image preprocessing}

\begin{itemize}
  \item Gaussian blur
  \item Turning the image into a greyscale (Finding the "brightest" part of the RGB, black and white)
  \item Adjusting the contrast of the image using Contrast Limited Adaptive Histogram Equalization
  \item Further brighten colorful areas to better separate them from black areas
  \item Converting the image into a binary image via a binary threshold
  \item Using the image morphology erosion to increase the size of black areas
  \item Turn white rectangles in white ones with black borders (via a mask and a dilation)
\end{itemize}

The goal of the image preprocessing step is to turn the input image into a
black-and-white or binary image where the lines of a Mondrian paiting are black
while the inner side of the rectangles are white.

At first the image is transformed with a minimal Gaussian blur to reduce
artifacts that the input images might include. (Reference Gaussian blur paper?)

After that the RGB input image is converted into a greyscale image by using the
maximum value of the RGB triplet.

Now the contrast of the image is harmonized using the Contrast Limited Adaptive
Histogram Equalization algorithm.

As the next step the colorful areas of the images are further brightened to make
them further stand out against the black areas of the paintings. This way darker
blue areas can be better distinguished compared to black areas.

Now the image is converted into a binary image using an arbitrary threshold that
can be configured.

Since this step will likely also turn a few areas white that a human would
recognize as black line an erosion is applied now. Erosion is an image morphology
that increases the size or black areas in an image. This way possible unwanted
cracks in the lines are restored.

The resulting image now separes the darker parts of the image fairly well.
However we are only interested into black lines, but Mondrian paitings also include
black rectangles. To remove these black rectangles we create a mask that applies a
comparatively strong dilation on the image (the oppotive of an erosion). The resulting
mask is then removed from the image in the step before. This way only the outline of the
black rectangles remains. (This could actually be shown as a formular, A - Erode(A))


\subsection{Detection and recognition of rectangles}




\begin{itemize}
  \item Find all horizontal and vertical lines by going through all the pixels
  \item Combine close parallel lines to one line
  \item Connect the lines so that the ends of each line always touch another line
  \item Find the four different kind of corners that these lines form
  \item Find the different rectangles given these corners
  \item Determine the color of the rectangles by comparing the area with the original
\end{itemize}}

% Siehe \zb \cite{Dje06, DjeOezSal07, CocWil00}.
