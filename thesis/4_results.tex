% !TeX encoding = UTF-8
\section{Results} \label{results}

 The image selection was restricted to paintings 1920-1937. There are 178 works
 listed in the \textit{Catalogue Raisonn{\'e}} \cite{joosten1998} for that time
 period. 28 of those were sketches, unfinished work or not a painting. Of the
 remaining 12 were drawn on a diamond-shaped canvas and for 20 paintings was
 either no or only a greyscale image available. The remaining 178 images were
 scanned and subsequently cropped so that only the painting and no frame was
 visible.  The images were scaled so that the longer side would be 1000
 pixels. The images were then processed by the detection program and subsequently
 checked for accuracy by comparing the result with the original. The resulting
 dataset consists of 1316 rectangles and their respective size, colors, and
 paintings.

Some explorative analysis about the use of color (\ref{color}) and ratios
(\ref{ratios}) was performed.

\subsection{Colors} \label{color}

On average Mondrian paintings consist of 79.7\%  and non-colors (black, white)
20.3\% colors (red, blue, yellow) ($\sigma = 17.0\%$), see Figure
\ref{fig:colors-noncolors}. The median percentage of each color appears to be
rather similar, the distribution of read however is more widespread as you can
see in Figure \ref{fig:colors-rby}.

\begin{figure}
\includegraphics[width=\linewidth]{images/colors-non-colors.png}
\caption{Percentage of color/non-color of the total area}
\label{fig:colors-noncolors}
\end{figure}

\begin{figure}
\includegraphics[width=\linewidth]{images/colors-rby.png}
\caption{Percentage of red, blue and yellow of the total area}
\label{fig:colors-rby}
\end{figure}

To see if different colors of rectangles have overall preferred positions in the
compositions, we assembled a dataset of all rectangles and their respective
positions, width and colors. We calculated the centers of all of these
rectangles and plotted them by color. We also visualized the estimated
probability density function of the positions using Kernel density estimation
with an Gaussian kernel selected by Scott's method \cite{Terrell1992}. The
result is given in Figure \ref{fig:kde}. The first thing you notice is that the
color red appears to have a substantial bias to the top-left corner. Similarly
blue has a slighter bias to the bottom-right corner.

\begin{figure}
\includegraphics[width=\linewidth]{images/kernel-densities.png}
\caption{Scatter plots and KDE for each color}
\label{fig:kde}
\end{figure}

\subsection{Ratios} \label{ratios}

For all rectangles we calculated the aspect ratio of the longer to the shorter
side. Figure \ref{fig:aspect-rects} shows the estimated probability density of
the ratios and Figure \ref{fig:longer-x-shorter} shows all rectangles plotted by
their sides as well as lines for certain proposed ratios. The data does not show
peaks for the golden or silver ratio. This supports the rejection by Livio
(2002) \cite{Livio2002} and Markowsky (1992) \cite{Markowsky1992} of the golden
rectangle claim by Bouleau (1963) \cite{bouleau1963} and Bergamini (1980)
\cite{bergamini1980}.

\begin{figure}
\includegraphics[width=\linewidth]{images/aspect-max-min-rects.png}
\caption{Rectangles ratio KDE}
\label{fig:aspect-rects}
\end{figure}

\begin{figure}
\includegraphics[width=\linewidth]{images/longer-x-shorter.png}
\caption{Rectangles longer side to shorter side}
\label{fig:longer-x-shorter}
\end{figure}


% TODO: Number of rectangles over time?
% TODO: How many paintings use all colors? How many with only one?
