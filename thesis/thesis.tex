% !TeX encoding = UTF-8

%===============================================================================
% Font options are:
%   plain (default), serif (uses Palladio), sans-serif (uses Paratype Sans)
% Layout options are:
%   article (default, no chapters), book (for longer texts, offers \chapter)
% Paragraph options are:
%   noparskip (default, no spacing between paragraphs), parskip (spaced)
\documentclass[serif,article,noparskip]{agse-thesis}

% Global parameters, replace with actual values.
\newcommand{\thesisTitle}{Über den Sinn des Lebens}
% -> You may use \par (but not \\) to format the title. If you do so, you'll
%    need to manually set the 'pdftitle' attribute below.
\newcommand{\studentName}{Finn Pauls}
%===============================================================================

\hypersetup{pdftitle={\thesisTitle}}
\hypersetup{pdfauthor={\studentName}}

% Blind texts, for demonstration only, not part of the actual template
\usepackage{lipsum}

\begin{document}

\coverpage[
    student/id=1234567,
    student/mail=finn@inf.fu-berlin.de,
    thesis/type=Bachelorarbeit,            % optional, default: Bachelorarbeit
    thesis/group={Arbeitsgruppe Software Engineering},
                                           % optional, default: AGSE
    thesis/advisor={Matt Visor},           % optional
    thesis/examiner={Prof. Dr. Mia Maus},
    thesis/examiner/2={Prof. Dr. Bob Bär}, % optional
    thesis/date=\today,                    % optional, default: \today
   %title/size=\LARGE,      % set this value to overwrite automatic font size
   %abstract/separate       % toggle this to move the abstract to its own page
]
{ % Your abstract here:
    \lipsum[1]
}

% !TeX encoding = UTF-8
\subsection*{Eidesstattliche Erklärung}

Ich versichere hiermit an Eides Statt, dass diese Arbeit von niemand anderem
als meiner Person verfasst worden ist. Alle verwendeten Hilfsmittel wie
Berichte, Bücher, Internetseiten oder ähnliches sind im Literaturverzeichnis
angegeben, Zitate aus fremden Arbeiten sind als solche kenntlich gemacht. Die
Arbeit wurde bisher in gleicher oder ähnlicher Form keiner anderen
Prüfungskommission vorgelegt und auch nicht veröffentlicht.\\

\thesisDate \\

\studentName


\cleardoublepage

\tableofcontents

\cleardoublepage

\mainmatter

% !TeX encoding = UTF-8
\section{Einführung}

Das Einführungskapitel beinhaltet ein paar praktische Hinweise zum Schreiben
der Abschlussarbeit, sowie eine Kurzdokumentation der bereitgestellten
\LaTeX-Klasse.
Die restlichen Kapitel dienen lediglich zu Demonstrationszwecken.

\subsection{Zur Abschlussarbeit als solche}

Neben der jeweiligen Studien- und Prüfungsordnung, die die förmlichen
Eigenschaften der Durchführung einer Abschlussarbeit regelt, sind folgende
Quellen hilfreich:
\begin{itemize}
    \item Studien- und Prüfungsordnungen der Informatikstudiengänge
    (\url{http://www.imp.fu-berlin.de/fbv/pruefungsbuero/Studien--und-Pruefungsordnungen/index.html})

    \item ThesisRules (\url{http://www.inf.fu-berlin.de/w/SE/ThesisRules}):

    Beschreibung des praktischen Ablaufs einer Abschlussarbeit in der AG
    Software Engineering von A bis Z.

    \item "`Technisches Schreiben"'
    (\url{http://www.mi.fu-berlin.de/wiki/pub/SE/SeminarRegeln/Technisches_Schreiben.pdf}):

    Ein von Lutz Prechelt verfasstes Dokument mit vielen praktischen
    Hinweisen zum Schreibteil (nicht nur) einer Abschlussarbeit.
\end{itemize}

\subsection{Zu dieser \LaTeX{}-Vorlage}

\subsubsection{Optionen der Dokumentenklasse}

Die Dokumentenklasse \texttt{agse-thesis} unterstützt verschiedene
Schriftarten:
\begin{lstlisting}[language={[LaTeX]TeX}]
% Standard LaTeX Schriftart
\documentclass[plain]{agse-thesis}

% Serifenschrift Palladino
\documentclass[serif]{agse-thesis}

% Serifenlose Schrift Paratype Sans
\documentclass[sans-serif]{agse-thesis}
\end{lstlisting}

Für kürzere Arbeiten, die mit Abschnitten (\texttt{\textbackslash{}section})
als oberste Gliederungsebene auskommen, reicht die Standard-Option
\texttt{article}.
Die Buch-Option \texttt{book} bietet darüber hinaus noch Kapitel
(\texttt{\textbackslash{}chapter}) an.
\begin{lstlisting}[language={[LaTeX]TeX}]
% Standard fuer kuerzere Arbeiten
\documentclass[article]{agse-thesis}

% Buch-Variante fuer umfangreiche Arbeiten mit vielen
% Gliederungselementen
\documentclass[book]{agse-thesis}
\end{lstlisting}

Ob zwischen den Absätzen im Text Abstände angezeigt werden sollen, oder ob
stattdessen die erste Zeile eines Absatzes eingerückt werden soll, kann mit
\texttt{parskip} bzw. \texttt{noparskip} eingestellt werden.
\begin{lstlisting}[language={[LaTeX]TeX}]
% Absaetze deutlich trennen
\documentclass[parskip]{agse-thesis}

% Absaetze nah bei einander, erste Zeile eingerueckt
\documentclass[noparskip]{agse-thesis}
\end{lstlisting}

Die Werte der drei o.g. Optionen können beliebig kombiniert werden:
\begin{lstlisting}[language={[LaTeX]TeX}]
% Einstellung des Beispieldokuments
\documentclass[serif,article,noparskip]{agse-thesis}
\end{lstlisting}

\subsubsection{Befehl \texttt{\textbackslash{}thesisTitle}}

Der Titel der Arbeit wird sowohl auf der Titelseite (siehe
\ref{sec:cmd-coverpage}) als auch für die PDF-Metainformationen benötigt.
Gesetzt wird der Titel durch das Definieren von
\texttt{\textbackslash{}thesisTitle}.

Für die Titelseite können manuell mit \texttt{\textbackslash{}par}
Zeilenumbrüche eingefügt werden um das Textbild zu verbessern (nicht hingegen
mit \texttt{\textbackslash\textbackslash}).
Sollte von dieser Möglichkeit Gebrauch gemacht werden, muss der Titel für die
PDF-Metainformationen manuell gesetzt werden
(\texttt{\textbackslash{}hypersetup\{pdftitle=\{...\}\}}).


\subsubsection{Befehl \texttt{\textbackslash{}coverpage}}
\label{sec:cmd-coverpage}

Die Titelseite der Abschlussarbeit wird mit dem
\texttt{\textbackslash{}coverpage}-Befehl erzeugt.
Dessen Ausgabe wird über eine Reihe von Schlüssel-Wert-Paaren konfiguriert
(siehe \autoref{tab:coverpage-config}).
\begin{table}[h]
\begin{center}
    \begin{tabular}{|l|L{5.5cm}|L{4cm}|}
        \hline
        \textbf{Schlüssel} & \textbf{Funktion} & \textbf{Default-Wert} \\
        \hline
        \texttt{student/id} & Matrikel-Nummer & -- \\
        \texttt{student/mail} & E-Mail-Adresse & -- \\
        \texttt{thesis/type} & Art der Abschlussarbeit & "`Bachelorarbeit"' \\
        \texttt{thesis/group} & Arbeitsgruppe in der die Arbeit geschrieben
        wurde & "`Arbeitsgruppe Software Engineering"' \\
        \texttt{thesis/advisor} & \emph{optional:} Betreuer der Abschlussarbeit
        & -- \\
        \texttt{thesis/examiner} & Erstgutachter der Arbeit & -- \\
        \texttt{thesis/examiner/2} & \emph{optional:} Zweitgutachter der Arbeit
        & -- \\
        \texttt{thesis/date} & \emph{optional:} Datum der Abgabe & aktuelles
        Datum\\
        \texttt{title/size} & \emph{optional:} \LaTeX-Schriftgröße für den
        Titel (\zb \texttt{\textbackslash{}LARGE}) & wird automatisch gesetzt \\
        \texttt{abstract/separate} & \emph{optional:} Schlüssel ohne Wert;
        falls gesetzt, wird der Abstract auf eine eigene Seite gesetzt und die
        Titelseite ist "`luftiger"' & -- \\
        \hline
    \end{tabular}
    \caption{Schlüssel-Wert-Konfiguration des
    \texttt{\textbackslash{}coverpage}-Kommandos.}
    \label{tab:coverpage-config}
\end{center}
\end{table}
Das einzige Argument des Kommandos ist der Abstract der Arbeit.
Ein minimaler Aufruf könnte so aussehen:
\begin{lstlisting}[language={[LaTeX]TeX}, morekeywords={coverpage}]
\coverpage[
    student/id=1234567,
    student/mail=email@inf.fu-berlin.de,
    thesis/type=Masterarbeit,
    thesis/examiner={Prof. Dr. Mia Maus}
]
{
    Prokrastination ist ein gut verstandenes Verhalten,
    das auch vor Abschlussarbeitern mit Informatik-Hintergrund
    nicht halt macht.
    % ...
}
\end{lstlisting}


\subsubsection{Verbesserungen der \LaTeX-Vorlage}

Diese \LaTeX-Vorlage soll den Einstieg in das Setzen der Abschlussarbeit
erleichtern.
Die Vorlage selbst wird in einem öffentlichen Git-Repository in der
GitLab-Instanz des Fachbereiches verwaltet, welches gerne als Grundlage für die
eigene Ausarbeitung geklont werden darf:
\begin{lstlisting}[language=bash]
git clone https://git.imp.fu-berlin.de/agse/thesis-template
\end{lstlisting}
Änderungsvorschläge in Form von Merge-Requests sind jederzeit willkommen.

% !TeX encoding = UTF-8

\section{Methods}

\subsection{Concepts}

For preprocessing the image several standard methods are used: \textit{Image
Morphology}, \textit{Thresholding}, \textit{Contrast Limited Adaptive Histogram
Equalization (CLAHE)} and Masks. These algorithms are provided by Python
distribution of the Open Source library OpenCV \cite{opencv_library}.

\subsection{Image Morphology}

Image Morphology is based on Mathematical Morphology from the mathematical field
of set theory. We are going to present two basic morphological operators
commonly used in image processing without going into detail of their mathematical
definition.

Given a binary image $X$ and a smaller binary image $B$ called the
\textit{Structuring Element}, we can apply the following operations.

The Erosion $X \ominus B$ is equivalent to iterating through all of the white
pixels of $X$ and only keeping them if $B$ translated to this position is fully
included in $X$. Therefore the white areas in $X$ are reduced \cite{Smith1997}.

Similarly with the Dilation operation $X \oplus B$ neighbouring pixels are
turned white if they intersect with the translated Structuring Element. The size
of white areas is increased.

Image Morphologies have multiple uses like increasing certain areas for better
recognition or separating larger shapes of an image.

% Concepts used
% - Thresholding gray-scale images
% - Contrast Limited Adaptive Histogram Equalization (CLAHE)
% - Masks

\subsection{Implementation}


The input to the detection algorithm is an RGB image of a Mondrian painting. The
expected output is a list of rectangles with their position, sizes and color as
they are conceptionally seen in the painting. All rectangles combined are
expected to exactly cover the area of the input image in a way that the original
could be dissected into this set of rectangles.

The algorithm itself can be separated into two different phases: Image
preprocessing and the recognition of rectangles.

The goal of the first preprocessing phase is to get a binary image (an image
with only black and white pixels) in which the black represent the lines of the
painting while the white represents the inner area of the rectangles.

The second phase then takes this binary image and returns a list of found
rectangles. Additionally, for recognizing the colors of the rectangles, the
original input image is used.

\subsubsection{Image preprocessing}

At first, the input image is transformed with a Gaussian blur to reduce
artifacts that the input images might include.

Next, the RGB input image is decomposed into a greyscale image by always using
the maximum value of the RGB triplet for each pixel.

To normalize the greyscale image and increase the contrast of the darker
regions, we now apply histogram equalization. Since global equalization did not
work well for all images, we instead used Contrast Limited Adaptive Histogram
Equalization (CLAHE) from OpenCV. In contrast to conventional adaptive histogram
equalization algorithms, this prevents the overamplification of noise.

After the normalization is applied, we now further brighten colorful areas of
the image to make them further stand out against the black areas of the
paintings. This way, for example, darker blue areas can be better distinguished
from black areas. Since the difference between the maximum and the minimum value
of the RGB triplet is related to the colorfulness of a color, we calculate it
for every pixel of the input image. The resulting mask from this calculation is
added to the contrast-normalized image.

Now the image is converted into a binary image $A_1$ using an arbitrary
threshold for the greyscale values. This value of the threshold can be
configured. Through test runs a value of 110 for a greyscale range of 0 to 255
showed to give the reasonable results.

Since the thresholding step might leave some interruption in the black lines, we
reduce the white areas of the image: An Erosion $A_2 = A_1 \ominus B_e$ is
applied. In this case, the structuring element $B_e$ is a $N\times N$ filled
matrix. Using the Erosion some accidental interruptions in the lines can be
restored. However choosing a the Structuring Element that is too high, might
result into loss of information. Therefore the size $N$ needs to be chosen
carefully.

The resulting image $A_2$ now separates the darker parts of the image fairly
well. However, we are only interested in black lines, but Mondrian paintings
also include filled black rectangles. To remove these black rectangles, we
create a mask that applies a Dilation $A_m = A_2 \oplus B_d$ with a
comparatively larger Structuring Element $B_d$ on the image. The size of $B_d$
is chosen in a way that most of the lines in the paintings are removed, ideally
only leaving inner areas of black rectangles. The resulting mask $A_m$ is then
removed from the image of the last step $A_3 = A_2 - A_m$. Hence only the
outlines of the black rectangles, as well as the black lines, remain in $A_3$.

\subsubsection{Detection and recognition of rectangles}

Since the rectangles in the image are defined through the horizontal and
vertical lines in the image, the detection starts by finding all of those lines
in the output binary image from the previous phase.

To find horizontal lines, the image is scanned line by line for each pixel.
Uninterrupted sequences of black pixels are added to a list of horizontal lines.
Only sequences with a specified minimum length are added. The minimum length
should ideally be slightly larger than the maximum width of the lines in the
image. Vertical lines are recognized respectively. This step is the equivalent of
applying a lossy run-length encoding once horizontally and once vertically.

However, this means that what we perceive as one horizontal line in the painting
is now recognized as multiple lines next to each other. For example, a line that
is 50 pixels wide would be recognized as 50 lines. Therefore parallel lines
close to each other are merged into a single line. As long as parallel lines are
within a certain distance from each other, they are merged into one line. For
horizontal lines, the resulting line will have the average $y$ value of all
those lines and the minimum and maximum $x$ values as starting and end points.

From out idealistic view on Mondrian paintings, the ends of lines always touch
another line or the edge of the painting. However, the lines that we reduced now
might overlap slightly or not even connect to the next line.

We, therefore, iterate over all the ends of the lines, determine the closest
line to this point and change the position of this end to the closest line. For
the purpose of this algorithm, the edges of the image are considered to be lines
as well.

All lines should now represent the structure of the painting. What we want as
our output though is a list of rectangles. Every rectangle in the image can be
defined through a set of four different corners: top-left, top-right,
bottom-left and bottom-right. These corners are always intersections of two
lines, either crossing or touching. We determine the corners and their types by
iterating through the lines.

After this, four different corners are combined into a rectangle by finding
matching corners. This is done by iterating through the top-left corners $(x,y)$
and finding the closest top-right corner to the right $(x_r, y_r)$ and the
closet bottom-left corner below $(x_b, y_b)$. The rectangle is then defined by
the position of the top-left corner and its width and height
$(x,y,x_r-x,y_b-y)$.

For this list of rectangles, we determine the colors of the rectangles. For this
purpose, we clip the rectangle from the original image and determine the average
color or the selection. This color is then reduced to either black, white, red,
blue or yellow.

% !TeX encoding = UTF-8
\section{Hauptteil}

Der folgende Programmcode ist nicht repräsentativ für das Ergebnis einer
erfolgreichen Abschlussarbeit.

\begin{lstlisting}
public class Main {
    public static void main(String[] args) {
        System.out.println("Hello World");
    }
}
\end{lstlisting}

% !TeX encoding = UTF-8
\section{Zusammenfassung}

...


\bibliography{bibliography}

\appendix
% !TeX encoding = UTF-8
\section{Anhang}

Quellcode der \LaTeX-Klasse \texttt{agse-thesis}:\footnote{Es ist nicht üblich,
den gesamten produzierten Quellcode bei einer Abschlussarbeit in Textform
abzugeben.}

\lstinputlisting[
    language={[LaTeX]Tex},
    morekeywords={ProvidesClass, DeclareOption, PassOptionsToClass,
        ProcessOptions, CurrentOption, LoadClass, RequirePackage, ifthenelse,
        ifcsdef, equal, definecolor, lstset, pgfkeys},
    basicstyle=\footnotesize\ttfamily,
    numbers=left,
    numberstyle=\footnotesize\ttfamily,
    stepnumber=5,
    inputencoding=utf8,
    extendedchars=true,
   literate={ä}{{\"a}}1 {ü}{{\"u}}1,
]{agse-thesis.cls}


\end{document}
