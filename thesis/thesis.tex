% !TeX encoding = UTF-f

\documentclass[serif,article,noparskip]{agse-thesis}
\usepackage{amsmath}
\usepackage{subcaption}

\newcommand{\thesisTitle}{The Rhythm of Relations of Color and Size: Detecting
Rectangles in Paintings Created by Piet Mondrian}
% -> You may use \par (but not \\) to format the title. If you do so, you'll
%    need to manually set the 'pdftitle' attribute below.
\newcommand{\studentName}{Finn Pauls}
%===============================================================================

\hypersetup{pdftitle={\thesisTitle}}
\hypersetup{pdfauthor={\studentName}}

\begin{document}

\coverpage[
    student/id=4788442,
    student/mail=finn.pauls@fu-berlin.de,
    thesis/type=Bachelorarbeit,            % optional, default: Bachelorarbeit
    thesis/group={Arbeitsgruppe Theoretische Informatik},
    thesis/advisor={M.Sc. Martin Skrodzki},
    thesis/examiner={Prof. Dr. Wolfgang Mulzer},
    thesis/examiner/2={Prof. Dr. Konrad Polthier},
    thesis/date={1. Oktober 2018},                    % optional, default: \today
   %title/size=\LARGE,      % set this value to overwrite automatic font size
   % abstract/separate       % toggle this to move the abstract to its own page
] { The arrangement of compositional elements in abstract paintings by Piet
Mondrian have been of interest to researchers for a long time. Despite this
interest, there are no publications examining these paintings numerically across
a broad pool of data. This thesis tries to change this lack of data by
describing an algorithm for detecting rectangles in Mondrian's compositions from
1920 to 1937. It also provides a descriptive analysis of the resulting data.
While there are no hints that Mondrian used specific aspect ratios, a bias for
positioning colors could be revealed.}

% !TeX encoding = UTF-8
\subsection*{Eidesstattliche Erklärung}

Ich versichere hiermit an Eides Statt, dass diese Arbeit von niemand anderem
als meiner Person verfasst worden ist. Alle verwendeten Hilfsmittel wie
Berichte, Bücher, Internetseiten oder ähnliches sind im Literaturverzeichnis
angegeben, Zitate aus fremden Arbeiten sind als solche kenntlich gemacht. Die
Arbeit wurde bisher in gleicher oder ähnlicher Form keiner anderen
Prüfungskommission vorgelegt und auch nicht veröffentlicht.\\

\thesisDate \\

\studentName


\cleardoublepage

\tableofcontents

\cleardoublepage

\mainmatter

\section{Introduction}

\begin{figure}
\centering
\begin{subfigure}{.3\textwidth}
  \centering
  \includegraphics[width=\linewidth]{images/B104.jpg}
  \caption{Composition II (1920)}
  \label{fig:sub1}
\end{subfigure}%
\begin{subfigure}{.3\textwidth}
  \centering
  \includegraphics[width=\linewidth]{images/B123.jpg}
  \caption{Composition with Red, Blue and Yellow (1930)}
  \label{fig:sub2}
\end{subfigure}
\begin{subfigure}{.3\textwidth}
  \centering
  \includegraphics[width=\linewidth]{images/B273.jpg}
  \caption{Opposition of Lines: Red and Yellow (1937)}
  \label{fig:sub3}
\end{subfigure}
\caption{Three examples of Mondrian paintings from 1920, 1930 and 1937.}
\label{fig:mondrian}
\end{figure}

\subsection{Piet Mondrian}

The oeuvre of Piet Mondrian (1872-1944) consists of figurative paintings as well
as abstract compositions. Up until 1910, his works only depicted naturalistic
scenes like churches, trees, windmills or landscapes. From 1911 onwards his
paintings remain representational but are increasingly painted in a more
abstract way.

Around 1914, his compositions begin to be purely abstract and in 1917 he cofounds
the artistic movement and group \textit{De Stijl}, also known as
\textit{Neoplasticism}. One of the primary objectives of the group is to reform
art by "abolishing natural form" \cite{wiki:manifest} altogether.

At some point around 1920, Mondrian further restricted his compositional
elements to rectangles and straight black lines. He only allowed lines to run
horizontally or vertically and only used primary colors (red, blue and yellow)
and non-colors (black, white and grey). See Figure \ref{fig:mondrian} for
examples of Mondrian paintings with these elements.

Another founding member of \textit{De Stijl}, Georges Vantongerloo, used methods
inspired by mathematics for his compositions. One example of these paintings is
\textit{Composition Derived from the Equation y = -ax2 + bx + 18}.

Mondrian, on the other hand, is known to only use intuition when creating his
works, moving compositional elements around for weeks, seeking a balanced
composition, until he was satisfied with the result.

Still, many people have been interested in the question if Mondrian's
non-figurative paintings utilize certain compositional rules or techniques that
the painter might have used unconsciously. One approach to finding structure in
Mondrian's paintings is to create art that resembles his compositions
using computers.

\subsection{Related Works}

There have been multiple attempts at creating art that resembles Mondrian's
compositions using computers. Feijs (2004) \cite{Feijs2004} presents different
techniques for generating images resembling non-figurative Mondrian paintings
from different periods. He concludes that different kinds of algorithms for
generating images can be used to formalize the distinction between different
types of Mondrian paintings.

Skrodzki and Polthier (2018) \cite{Skrodzki2018} use computer models to generate
Mondrian-inspired three-dimensional pieces using KdTree data structures. They
note however that while their results are similar to Mondrian paintings, they do
not always resemble Mondrian paintings since proportions are chosen randomly and
not deliberately.

In 1968 Hill \cite{Hill1968} tried to study the organization of the compositional
elements used by Mondrian. He analyzed the network topology of Mondrian's
paintings. One of his findings shows the avoidance of symmetry on a structural
level. He also criticized earlier work for the inaccuracy of measurement and the
lack of statistics and called for a more substantiated analysis of the
paintings.

Some sources \cite{bouleau1963,bergamini1980} suggest that Mondrian used golden
rectangles in his paintings. A golden rectangle is a rectangle where the ratio
between the sides lengths is the golden ratio $\phi \approx 1.618$. Other authors
 \cite{Livio2002,Markowsky1992} however refute those claims, finding fault with
the lack of evidence, which consists of mostly exemplary superimposing golden
rectangles on paintings. Despite this disagreement, there have been very few
attempts to actually evaluate this question statistically. In a poster from
2017, Tanaka and Miyanaga \cite{Tanaka2017} examine the use of rectangle ratios,
specifically the golden and silver ratio, in 10 late Mondrian paintings. They
conclude that Mondrian actually had a preference for the silver ratio $\delta_S
\approx 2.414$ instead.

However, to the author's knowledge, there is no publication examining the question of
the use of certain special ratios on Mondrian's paintings with a larger data
set.

\subsection{Scope of the Thesis}

Providing numeric data for closing the lack of foundation for analysis on the
neo-plastic compositions by Mondrian is the main objective of this thesis. The
goal is to create a program that uses methods of Computer Vision to extract the
compositional elements from images of the paintings. The program is restricted
to paintings from 1920 to 1937, which all use black lines to separate rectangles.

In order to have a uniform and exhaustive representation of the works in
question, the \textit{Catalogue Raisonn{\'e}} \cite{joosten1998} was chosen as a
source. It includes images of the complete works of Piet Mondrian together with
unique identifiers. For the works of  Piet Mondrian, this is particularly
important as many works are named very similarly or even have the same name.

There are 178 works listed in the catalog for that time period. Of these works,
150 were paintings while the other 28 were sketches, unfinished works or plastic
art. For 12 paintings, Mondrian used a tilted canvas, often referred to as
\textit{Diamond Compositions}. For the sake of simplicity, these paintings were
excluded as well. Additionally, 20 paintings in the catalogue had either only a
greyscale image or no image available at all. These paintings were also
excluded. The remaining 118 images were selected for the algorithm. They were
scanned and subsequently cropped so that only the painting and no frame was
visible. The images were scaled so that the longer side would be 1000 pixels.

Since the distinction between lighter grey to white and darker grey to black is
difficult even for a human observer, the detection of non-colors is restricted
to white and black only. Additionally, the lines are simplified to not have any
thickness, so that they can be represented solely by their start and end points.

\section{Fundamentals} \label{fundamentals}

This section will specify the different definitions used for conceptualizing the
program (\ref{definitions}). It is also describing different existing methods
and algorithms that were used (\ref{used}). Lastly, it outlines the concept for
detecting the structure of rectangles in Mondrian's abstract paintings
(\ref{concept}).

\subsection{Definitions} \label{definitions}

An Image $X$ with width $w$ and height $h$ is defined to be the set $I$. $$I =
\{p_{xy} | x \in \{1,2, \dots w\}, y \in \{1,2, \dots h\} \}$$ An RGB Image $I$
is an Image with the following  $p_{xy}$: $$p_{xy} = (r_{xy}, g_{xy}, b_{xy})
\in \{0,1,...,255\} \times \{0,1,...,255\} \times\{0,1,...,255\}$$ A Greyscale
image is an Image with $p_{xy} \in \{0,1, \dots 255\}$ and a Binary Image $B$ is
an Image with $p_{xy} \in \{0,1\}$, where $0$ is also referred to as
\textit{black} and $1$ as \textit{white}.

\subsection{Used Algorithms and Methods} \label{used}

For preprocessing the images, a combination of methods are used: \textit{Image
Morphology}, \textit{Thresholding}, \textit{Contrast Limited Adaptive Histogram
Equalization (CLAHE)} and \textit{Masks}. These algorithms are provided by the
Python distribution of the Open Source library OpenCV \cite{opencv_library}.

\subsubsection{Image Morphology}

Image Morphology is based on Mathematical Morphology from the mathematical field
of set theory. For the purpose of this algorithm, two basic morphological
operators are used. They are going to be explained briefly here, without
going into detail of their mathematical definition. A more formal definition and
further reading can be found in Aguardo (2012) \cite{Aguardo2012}.

Given a binary image $B$ and a smaller binary image $B_s$ called the
\textit{Structuring Element}, with a dedicated center point $c \in B_s$, two
basic operations can be applied: Erosion and Dilation. Figure \ref{fig:morphology}
illustrates the two operations on a $150 \times 150$ pixel example image with a
a Structuring Element of $10 \times 10$.

\begin{figure}
\centering
\begin{subfigure}{.3\textwidth}
  \centering
  \includegraphics[width=\linewidth]{images/example.jpg}
  \caption{Input}
  \label{fig:input}
\end{subfigure}%
\begin{subfigure}{.3\textwidth}
  \centering
  \includegraphics[width=\linewidth]{images/example-erode.jpg}
  \caption{Erosion}
  \label{fig:erosion}
\end{subfigure}
\begin{subfigure}{.3\textwidth}
  \centering
  \includegraphics[width=\linewidth]{images/example-dilate.jpg}
  \caption{Dilation}
  \label{fig:dilation}
\end{subfigure}
\caption{Demonstration of the Erosion and Dilation operation on an example input image
using a quadratic Structuring Element.}
\label{fig:morphology}
\end{figure}

\textit{Erosion} $B \ominus B_s$ is equivalent to going through all of the white
pixels of $B$ and translate $B_s$ by its center point $c \in B_s$ to the pixel's
position. The pixel is only kept white if all of the translated points of $B_s$
correspond to white pixels in the image. Otherwise, the pixel is turned black. Therefore
the white areas in $B$ are reduced. \cite{Smith1997}

Similarly the \textit{Dilation} operation  $B \oplus B_s$ translates the center
point of the Structuring Element to every pixel $p \in B$. For each of these
pixels, all the neighboring pixels that intersect with the translated $B_s$ are
turned white. The total area of white pixels is increased. \cite{Smith1997}

Image Morphologies have multiple uses like increasing certain areas for better
recognition or separating larger shapes of an image. For this thesis, they will
be helpful for correcting interruptions in the black lines of the image as well
as for separating black areas that are bigger than typical lines.

\subsubsection{Thresholding}

Thresholding $T(G) = \{t_{xy}\ \in \{0,1\} | g_{xy} \in G\}$ is a process for turning a greyscale image $G$
with pixels $g_{xy}$ into a binary image $B$ using a threshold $t \in \{
0,1,\dots255\}$. Each pixel $t_{xy}$ is assigned a black or white value given
whether it is below or above the threshold.

\begin{equation}
  t_{xy} =
  \begin{cases}
    0 & \quad  \text{if}  \quad g_{xy} \leq t\\
    1  & \quad \text{if}  \quad g_{xy} > t
  \end{cases}
\end{equation}

It is usually used to separate out regions of interest in an image. For the
algorithm presented in this paper, these regions will be the black lines of the
painting.

\subsubsection{Contrast Limited Adaptive Histogram
Equalization (CLAHE)} \label{clahe}

A \textit{Histogram} of an image is the distribution of brightness values in an
image. \textit{Histogram equalization} is the process of adjusting the contrast
of an image using the Histogram. Global histogram equalization changes the
brightness of each pixel in the image using the overall histogram of the image.
\cite{ShapiroLindaG2001Cv}

\textit{Adaptive histogram equalization (AHE)}, on the other hand, only uses the
histogram of a specified area around any given pixel to adjust its brightness. It
therefore is able to increase the local contrast better.

However, \textit{AHE} tends to overamplify noise in an image. Fairly uniform
regions of the same brightness have high histogram peaks and are turned into
noisy patterns.  \textit{Contrast Limited Adaptive Histogram Equalization
(CLAHE)} limits this effect by setting a maximum value for the values of the
histogram. \cite{Pizer1987}

\subsubsection{Masks}

Mask operations are related to elementary arithmetics. Given two images $X_a$,
$X_b$ with the same width and the same height, each pixel $a_{xy}$ in $X_a$ is
added, substracted, multiplied or divided with $b_{xy}$ of $X_b$. For example
the operation $X_a + X_b$ denotes the set $$X_a + X_b = \{\mathrm{min}(255,
p_{xy} + q_{xy}) | p_{xy} \in X_a, q_{xy} \in X_b\}.$$

An additional operation on a binary image $B_a$ is the opposite or inversion
$\neg B_a$, which is equivalent to a bitwise not.
$$\neg B_a = \{\mathrm{max}(0,p_{xy} - 1) | p_{xy} \in B_a\}$$

These operations will be helpful in combining results from different parts of
the algorithm or removing certain areas like black rectangles because later
steps should ignore them.

\subsection{Concept} \label{concept}

The aim for the detection algorithm is to take a cropped Mondrian painting as
an input and detect all the rectangles and their colors in it.

The input is an RGB Image $I_1$ of a Mondrian painting. The expected output is a
list of rectangles with their position, size, and color as they are
compositionally seen in the painting. All rectangles combined are expected to
completely dissect the original image. The thick black lines in the paintings are
seen as one-dimensional line segments that are positioned in the middle of the
lines in the painting.

The algorithm itself can be separated into two different phases: Image
preprocessing (\ref{preprocessing}) and recognition of rectangles (\ref{rectangles}).

The goal of the first preprocessing phase is to obtain a binary image in which
black represents the lines of the painting while white represents the inner area
of the rectangles. The different steps of the preprocessing phase are visualized
in Figure \ref{fig:preprocessing}.

The second phase then takes this binary image and returns a list of found
rectangles. Additionally, for recognizing the colors of the rectangles, the
original input image is used.

\subsubsection{Image Preprocessing} \label{preprocessing}

\begin{figure}
  \includegraphics[width=\linewidth]{images/preprocessing_steps.png}
  \caption{Preprocessing Steps on \textit{Composition II in Red, Blue, and Yellow
  from 1929}}
  \label{fig:preprocessing}
\end{figure}

At first, a Gaussian blur is applied to the input RGB Image $I_1$ to reduce
artifacts that the input images might include. The result of the blur is $I_2$.
These artifacts have several possible origins: They might be craquelure or
fading of darker areas in the paintings. But they might also be results of the
photography and scanning the images.

Next, the RGB input image is decomposed into a Greyscale Image $G_1$ by using
the maximum value of the RGB triplet for each pixel in $I_2$:
\begin{equation}
G_1 = \{p_{xy} \in \{0,\ldots,255\} \mid (r_{xy},g_{xy},b_{xy})\in I_2\text{ and }p_{xy}=\max\{r_{xy},g_{xy},b_{xy}\}\}.\\
\end{equation}

To normalize the brightness distribution of $G_1$ and increase the contrast of
the darker regions, histogram equalization is applied to $G_1$. For this we
use \textit{Contrast Limited Adaptive Histogram Equalization (CLAHE)}. In
contrast to non-adaptive and ordinary adaptive histogram equalization
algorithms, this method prevents the overamplification of noise as reasoned
above (\ref{clahe}). The result of the equalization is called $G_2$. \cite{Pizer1987}

After the normalization is applied, colorful areas of the image are further
brightened to make them stand out against the black areas of the paintings. This
way, for example, darker blue areas can be better distinguished from black
areas. Since the difference between the maximum values $G_1$ and the minimum
value $G_{min}$ of the RGB triplet is related to the colorfulness of a pixel, we
calculate the colorfulness for every pixel of the input image. The resulting
Mask $G_c$ from the calculation is added to the contrast-normalized image
$G_3$:
\begin{align}
G_{min} &= \{p_{xy} \in \{0,\ldots,255\} \mid (r_{xy},g_{xy},b_{xy})\in I_2\text{ and }p_{xy}=\min\{r_{xy},g_{xy},b_{xy}\}\}\\
G_c &= G_1 - G_{min}\\
G_3 &= G_2 + G_c
\end{align}

Now the image $G_3$ with the elements $g_{xy}$ is converted into a binary image
using Thresholding $B_1 = T(G_3)$. The threshold value $t$ needs to be chosen to
optimally separate between the darker lines and the lighter rectangles in the
image. See \ref{parameter} for how that value was determined in practice.

Since the thresholding step might leave some interruption in the black lines,
the white areas of the image are reduced: An Erosion $B_2 = B_1 \ominus B_e$ is
applied. In this case, the structuring element $B_e$ is a $N\times N$ white binary
Image. Using the Erosion some accidental interruptions in the lines can be
restored. However, choosing a Structuring Element that is too big might result
in loss of information by filling smaller white rectangles black. Therefore the
size $N$ needs to be chosen carefully.

The resulting image $B_2$ now separates the darker parts of the image fairly
well. But the goal is to detect black lines and Mondrian paintings also include
filled black rectangles. To remove these black rectangles, a mask is created
that applies a Dilation $B_m = B_2 \oplus B_d$ with a larger Structuring Element
$B_d$ ($M\times M$) compared to $B_e$ on the image. The size of $B_d$ is chosen
so most of the lines in the paintings are removed, only leaving inner areas of
black rectangles. The inverse of the resulting mask $B_m$ is then removed from
the image of the last step $B_3 = B_2 - B_m$. Hence only the outlines of the
black rectangles, as well as the black lines, remain in $B_3$, which is the
output of the first phase.

\subsubsection{Detection and Recognition of Rectangles} \label{rectangles}

Since the rectangles in the painting are defined by the horizontal and
vertical lines in the image, the detection starts by finding all of those lines
in the output binary image $B_3$ from the previous phase.

Now all uninterrupted vertical sequences $V$ and horizontal sequences $H$ of
black pixels with a minimum length $\ell$ are considered. You can think of it as
a lossy run-length encoding once horizontally and once vertically. The selection
of the parameter $\ell$ is discussed in \ref{parameter}.

This means that for every structural line in the painting, multiple line
segments are recognized. For example, a line that is 50 pixels wide would be
recognized as 50 lines. Therefore, as a next step, parallel lines close to each
other are merged into a single line. As long as parallel lines are within a
certain distance $d$ from each other, they are merged into one line. For
horizontal lines, the resulting line will have the average $y$ value of all
those lines and the minimum and maximum $x$ values as starting and end points.

From the conceptional view used in this thesis, the ends of lines always touch
another line or the edge of the painting. However, the remaining lines now might
overlap slightly or not even connect to the next line.

Now the two ends of every horizontal line $h \in H$ are considered. For each end
$e = (x_e,y_e)$ find the closest vertical line $v \in V$ that could touch $e$ by
only translating it horizontally.

Since a rectangle is defined by the black lines in the paintings, as well as the
edges of the painting, the lines of the edges to the $H$ and $V$ are added as
well.

All lines should now represent the structure of the painting. The desired output,
however, is a list of rectangles. Every rectangle in the image can be defined
through a set of four different corners: top-left, top-right, bottom-left and
bottom-right. These corners are always intersections of two lines, either fully
crossing or touching (T-crossing). The corners and their types can be
identified.

Next, four different corners are combined into a rectangle by finding matching
corners that belong to one rectangle. This is done by iterating through the
top-left corners $(x,y)$ and finding the closest top-right corner to the right
$(x_r, y)$ and the closest bottom-left corner below $(x, y_b)$. The rectangle is
then defined by the position of the top-left corner and its width and height
$(x,y,x_r-x,y_b-y)$.

The colors are determined for each rectangle in the list. For this purpose, the
rectangle is clipped from the original image $I_1$ and the average color of the
selection is calculated. This color is then reduced to either black, white, red,
blue or yellow.

\clearpage
\begin{lstlisting}[otherkeywords={function,input\:,output\:},label=lst:program,caption=Flow and structure of the program]
input: directory d,
    threshold t,
    erosion kernel size N,
    dilation kernel size M,
    minimum line length l,
    maximum line width d,

for every file f in d,
  read original image from file f
  binary = preprocessing(original)
  output = detect_rectangles(binary, original)
  check_image = create image from output and overlay it on original
  write check_image to disk
  write output to json file

function preprocessing
  input: RGB image of a Mondrian painting image_1
  output: Binary image where only the lines are black

  image_2 = blur(image_1)
  r,g,b = split_channels(image_2)
  grey_1 = max(r,g,b)
  grey_min = min(r,g,b)
  grey_2 = clahe(g1)
  grey_c = grey_1 - grey_min
  grey_3 = grey_2 + grey_c
  binary_1 = threshold(grey_3, t)
  binary_2 = erode(binary_1, N)
  binary_m = dilate(binary_1, M)
  binary_3 = binary_2 + bitwise_not(binary_m)
  return binary_3

function detect_rectangles
  input:
    Binary image where only the lines are black binary_image,
    Original RGB image of that painting original_img
  output: List of rectangles and their colors, dimensions, and positions

  (h1, v1) = detect_lines(binary_image, l)
  (h2, v2) = reduce_lines(h1, v1, d)
  (h3, v3) = remove_lines_close_to_border(h2, v2)
  (h4, v4) = add_border_lines(h3, v3)
  (h5, v5) = connect_lines(h4, v4)
  (tl, bl, br, tr) = find_corners(h5, v5)
  rects = find_rectangles(tl, bl, tr))
  rects_with_color = find_colors(rects, original_img)
  return rects_with_color
\end{lstlisting}

\section{Implementation} \label{implementation}

This section describes the implementation and development of the concept described
above. First, the structure of the software is presented (\ref{structure}),
then challenges in line detection (\ref{linedetection}) and performance
(\ref{performance}) are discussed. Also the iterative development approach
(\ref{iterative}) and the related selection of the parameters (\ref{parameter})
is explained. Lastly, the used tools are evaluated (\ref{evaluation}).

The program was implemented with a command line interface in \textit{Python 3.7}
using the \textit{OpenCV 3.4.1} and \textit{NumPy 1.15.2} libraries. OpenCV
provides the different image processing methods described in \ref{used} as well
as other methods for reading, writing and manipulating image files. NumPy, which
is part of the SciPy package, is a package for scientific numerical computing
\cite{scipy}.

\subsection{Structure} \label{structure}

The structure of the main algorithm could be closely modeled after the concept.
The separation between preprocessing (\ref{preprocessing}) and detection of the
rectangles (\ref{rectangles}) is also reasonable in the implementation.

The program sketched out in Listing \ref{lst:program} takes a directory and the
different parameters as input. The output of the main algorithm is a list of
rectangles and their respective dimensions, colors, and positions. An image is
created from the resulting data and overlaid over the original image and saved
to disk. This can be used to validate the result of the algorithm. The data from
the rectangles is encoded in a \textit{JSON} file for each image. \textit{JSON}
was chosen, since it is a popular flexible data-interchange format that is
also human-readable. An example for such a \textit{JSON} file can be seen in
Figure \ref{fig:json}.

\begin{figure}[b]
\begin{minipage}{0.2\textwidth}
  \includegraphics[width=\linewidth]{images/B244.jpg}
\end{minipage}
\begin{minipage}{0.8\textwidth}
\begin{lstlisting}
{
  "id": "B244",
  "height":1000,
  "width":489,
  "rectangles":[
    {"x":294,"y":91,"w":195,"h":691,"c":"white"},
    {"x":294,"y":782,"w":195,"h":218,"c":"white"},
    {"x":64,"y":0,"w":230,"h":1000,"c":"white"},
    {"x":294,"y":0,"w":195,"h":91,"c":"blue"},
    {"x":0,"y":0,"w":64,"h":1000,"c":"white"}
  ]
}
\end{lstlisting}
\end{minipage}
\caption{Source image and \textit{JSON} representation for \textit{Composition no. I (1934)}}
\label{fig:json}
\end{figure}

As an example, Listing \ref{lst:findrectangles} shows the source code of the
function \texttt{find\_rectangles}, a subroutine of \texttt{detect\_rectangles}.
It detects rectangles from a list of top-left, bottom-left and top-right
positions. The length of each of these lists is the same and is expected to be
the number of rectangles. The program first sorts the top-right corners by their
x position and the bottom-left corners by their y position. When iterating
through the top-left corners, their x position are going to be the same as the
matching bottom-left corner, apart from a larger y-coordinate. So only these
candidates are selected from the \texttt{b\_l} list using an iterator
expression. Since the corners are ordered from top to bottom, the matching corner
is going to be the first returned by the iterator. Therefore the iterator is
only called once with \texttt{next}. The same logic is applied to bottom-left
corners. Using the coordinates of the matching top-right and bottom-left
corners, the rectangle is calculated and added to a list of rectangles as a
tuple. That list is then returned by the function.

\begin{minipage}{\linewidth}
\begin{lstlisting}[otherkeywords=def,label=lst:findrectangles,caption=Function for constructing rectangles from corners]
def find_rectangles(t_l, b_l, t_r):
    t_r.sort(key=lambda pos: pos[0])
    b_l.sort(key=lambda pos: pos[1])
    rectangles = []
    for x,y in t_l:
        x2,_ = next(c for c in t_r if c[1] == y and c[0] > x)
        _,y2 = next(c for c in b_l if c[0] == x and c[1] > y)
        w = x2 - x
        h = y2 - y
        rectangles.append((x,y,w,h))
    return rectangles
\end{lstlisting}
\end{minipage}

\subsection{Line Detection} \label{linedetection}

Using a custom algorithm for the line detection was not the first choice. First, a common technique, the Hough Transform, was used. Hough Transforms rotate a
line over every pixel of an image with an angle $\theta$ for every step. For
each rotation black pixels on that lines are used as votes for that line. Then
lines above a certain threshold of votes are considered. Using $\theta =
\frac{\pi}{2}$ to only consider horizontal and vertical lines, this method did
not give accurate enough results. The algorithm would, for example, sometimes
detect a vertical line between two close horizontal ones. Since gaps between
lines are not expected after the preprocessing, this method is not suitable
in this context.

It became clear that it would be simpler and more accurate to construct the
lines by iterating through the image. To find horizontal lines $H$ from
\ref{rectangles}, the image is scanned vertically line by line. The
iteration runs through the pixels of each horizontal line. Uninterrupted
sequences of black pixels are added to a list of horizontal lines. Only
sequences with a specified minimum length $\ell$ are added. The minimum length
should be slightly larger than the maximum width of the lines in the image.
Vertical lines $V$ are recognized respectively by iterating through the lines
from left to right.

\subsection{Performance} \label{performance}

The performance of the algorithm was measured by timing the duration for each
image. The first implementation of the algorithm took about 1100 milliseconds per
iteration. By running timings on different parts of the processing, the
\texttt{detect\_lines} function was discovered as a bottleneck. It was
accountable for about 93 percent of the processing time.

This was the only place in the program were an iteration through all of the
pixels in the image took place using Python. All other operations were done
using OpenCV or NumPy. They also iterate through all of the pixels but using
optimized C code.

To reduce the time of this step this function was reimplemented in a compiled,
more low-level language. Rust was chosen using the \textit{rust-cpython} library
for bindings. Using the same algorithm implemented in Rust, the time of this
step was reduced from about 1020 to 40 milliseconds. The runtime of the program
could be reduced by a factor of 13.

This time could have been further reduced by running multiple parallel
iterations over the different rows or columns. However, for this purpose the
achieved time was sufficient.

\subsection{Iterative approach} \label{iterative}

\begin{figure}[b]
\centering
\begin{subfigure}{.23\textwidth}
  \includegraphics[width=\linewidth]{images/detect-1.jpg}
  \caption{\texttt{detect\_lines}}
  \label{fig:detect1}
\end{subfigure}%
\begin{subfigure}{.23\textwidth}
  \centering
  \includegraphics[width=\linewidth]{images/detect-2.jpg}
  \caption{\texttt{reduce\_lines}}
  \label{fig:detect2}
\end{subfigure}
\begin{subfigure}{.23\textwidth}
  \centering
  \includegraphics[width=\linewidth]{images/detect-3.jpg}
  \caption{\texttt{find\_corners}}
  \label{fig:detect3}
\end{subfigure}
\begin{subfigure}{.23\textwidth}
  \centering
  \includegraphics[width=\linewidth]{images/detect-4.jpg}
  \caption{\texttt{find\_colors}}
  \label{fig:detect4}
\end{subfigure}
\caption{Details of example output images with the corresponding function from Listing \ref{lst:program} after which they were created.}
\label{fig:detect}
\end{figure}

When developing the program, the different processing steps of the algorithm
were visualized by providing multiple output images for each input image. The
output images from the preprocessing phase correspond to the images shown in
Figure \ref{fig:preprocessing}. Example details of the same image from the
second phase can be seen in Figure \ref{fig:detect}. These images allow for an
inspection of the different intermediary results created in the process.
Together with the superimposed output image, these could be used to review the
correctness of the algorithm.

This visual feedback gave insights into problems of the algorithm like small
bugs or conceptional problems. For example, the preprocessing step to increase
the brightness of areas that are more colorful was introduced because there were
issues of discriminating darker blue areas from black ones. This way the
development of the algorithm could follow an iterative approach.

These images also provided a basis for the adjustment of the parameters of the
different steps.

\subsection{Parameter Selection} \label{parameter}

By running the program on the images and looking at the output images for the
different steps, the reasonable values for the parameters of the program can be
determined.

Through comparing the Thresholding output for each image to the input image, the
threshold value $t = 110$ could be determined to give reasonable results.

To better fine-tune the parameters, a subset of input images was selected and
the parameters manually changed until the result was correct. The resulting
\textit{JSON} files with the correct results were moved to another directory
\textit{detected}. Then a step was added to the program that would always
compare the computed result with the data in that directory if available. When
the result differed, it would print out a warning.

Using this output the minimum length $\ell$, the maximum width $d$ and the size
of the erosion kernel $N$ as well as the dilation kernel $M$ were changed to
maximize the number of recognized images from that set. The resulting parameters
were $l = 60$, $d = 70$, $N=11$ and $M =40$.

\subsection{Evaluation of Tools} \label{evaluation}

Reflecting about the used tools, OpenCV would certainly be chosen again. It
supports a variety of established computer vision algorithms, is well documented
and has a variety of learning resources. Since OpenCV primarily supports C++ and
Python interfaces, the programming language choice was one of these options.
While using the compiled language C++ could have increased performance over the
interpreted Python, it would also have lacked some of its benefits: Python's
readability and user-friendly data-structures made it a good choice for an
iterative development.

\section{Results} \label{results}

 All 118 images were processed by the detection program and subsequently checked
 for accuracy by comparing the result with the original. The resulting dataset
 consists of 1316 rectangles and their respective size, colors, and paintings.

Some descriptive analysis about the use of color (\ref{color}) and ratios
(\ref{ratios}) was performed.

\subsection{Colors} \label{color}

On average, neglecting the area used by lines, Mondrian paintings consist of
79.7\% non-colors (black, white) and 20.3\% colors (red, blue, yellow) ($\sigma
= 17.0\%$), see Figure \ref{fig:colors-noncolors}. While the median percentage
of each color appears to be rather similar, the distribution of red is more
widespread as can be seen in Figure \ref{fig:colors-rby}.

\begin{figure}
\centering
\begin{subfigure}{.5\textwidth}
  \includegraphics[width=\linewidth]{images/colors-non-colors.png}
  \caption{ }
  \label{fig:colors-noncolors}
\end{subfigure}%
\begin{subfigure}{.5\textwidth}
  \centering
  \includegraphics[width=\linewidth]{images/colors-rby.png}
  \caption{ }
  \label{fig:colors-rby}
\end{subfigure}
\caption{The preference for non-colors in the paintings is evident in (a).
The Median proportion of the colors in (b) are very similar, but red has a
wider distribution.}
\end{figure}

To see if different colors of rectangles have overall preferred positions in the
compositions, the dataset of all rectangles and their respective positions, width
and colors is used. The calculated centers of all of these rectangles are
plotted by color. The visualisation in Figure \ref{fig:kde} shows the estimated
probability density function of the positions using \textit{Kernel density
estimation (KDE)} with an Gaussian kernel selected by Scott's method
\cite{Terrell1992}. The first notable characteristic is that the color red
appears to have a substantial bias to the top-left corner. Similarly blue has a
bias to the bottom-right corner but is comparatively more evenly spread.


\begin{figure}
\includegraphics[width=\linewidth]{images/kernel-densities.png}
\caption{Scatter plots and KDE for each color}
\label{fig:kde}
\end{figure}

\subsection{Ratios} \label{ratios}

For all rectangles, the aspect ratio of the longer to the shorter side was
calculated. Figure \ref{fig:aspect-rects} shows the estimated probability
density of the ratios in Mondrian's rectangles compared to the ratio of a set of
10000 random rectangles. Figure \ref{fig:longer-x-shorter} shows all rectangles
plotted by their sides, as well as lines for certain proposed ratios. The data
does not show peaks for the golden or silver ratio. This supports the rejection
by Livio (2002) \cite{Livio2002} and Markowsky (1992) \cite{Markowsky1992} of
the golden rectangle claim by Bouleau (1963) \cite{bouleau1963} and Bergamini
(1980) \cite{bergamini1980}.

\begin{figure}
\centering
\begin{subfigure}{.5\textwidth}
  \includegraphics[width=\linewidth]{images/aspect-max-min-rects.png}
  \caption{ }
  \label{fig:aspect-rects}
\end{subfigure}%
\begin{subfigure}{.5\textwidth}
  \centering
  \includegraphics[width=\linewidth]{images/longer-x-shorter.png}
  \caption{ }
  \label{fig:longer-x-shorter}
\end{subfigure}
\caption{There appears to be no specific ratio used preferentially by Mondrian.}
\end{figure}


\section{Discussion} \label{conclusion}

The developed program is able to detect rectangles in a class of abstract
Mondrian paintings. Human control and intervention are still necessary to make
sure the result is correct. In 23 cases the input image was manually edited to
trace lines that were not detected. Otherwise, all images were detected
successfully by adjusting the input parameters. From the dataset of all chosen
images, 77\% of the images could be detected without changing the default
parameters.

A first look at the resulting data showed some interesting findings that deserves
further investigation. The dataset that was obtained from the images will be made
freely available for other applications, from statistical analysis to machine
learning.

The data does not suggest a preferred use of the golden or silver ratio in
rectangles. There only seems to be a general tendency towards squares.
Justifying continued research into this direction should depend on findings in
the psychology of aesthetics. So far the studies about the visually pleasing
nature of golden rectangles appear to contradict each other. Perceptional
studies about the silver ratios are missing entirely. However, research in  ratios
of other components, like the positioning of lines or the size of rectangles to
each other might lead to interesting results.

In "Art and Visual Perception" Rudolf Arnheim \cite{Arnheim1965} notes that "an
object in the upper part of the composition is heavier than none in the lower;
and location at the right side makes more weight than location on the left."
Winner (1987) \cite{Winner1987} psychologically tested these principles by
showing abstract images and their horizontally or vertically flipped
counterparts. Their findings support the up-down principle, but not the
left-right principle.

Arnheim also notes that the color red is heavier than the color blue. A study
from \cite{Locher2005} found some support for this claim when showing
design-trained participants modified Mondrian paintings where some colors were
swapped.

It would be interesting to explore if the found preference by Mondrian to use
red in the top-left corner could be explained by a combination of these proposed
principles of balance or whether it is merely coincidental or habitual.


% Do digital idealized representation of Mondrian paintings are comparable
% to the actual paintings in terms of certain factors like perceptional balance

% TODO: Big is heavier than small (i think)

% More generally: More paintings by Mondrian, other artists of De Stijl and
% artists in general

% Color distribution numbers close to Pareto's Principle. Maybe 80% of the
% effect of the painting stem from the 20% of the colors. Psychological research?
% eye-tracking studies

\bibliography{bibliography}

% \appendix
% \include{5_appendix}

\end{document}
