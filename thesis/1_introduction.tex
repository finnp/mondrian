% !TeX encoding = UTF-8
\section{Introduction}

Piet Mondrian (1872-1944) oeuvre consists of naturalistic painting as well
as abstract compositions. Up until 1910 his works depicted scenes like
churches, trees, windmills or landscapes. From 1911 his paintings
remain representational but start to be painted in a more abstract way.

Around 1914 his compositions begin to be purely abstract and in 1917 he founds
the artistic movement and group \textit{De Stijl} also known as
\textit{Neoplasticism}. One of the stated goals of this movement is to reform
art by "abolishing natural form" \cite{wiki:manifest} altogether.

% Maybe mention that they had a journal?

At some point then Mondrian restricted his compositional elements to rectangles
and horizontal as well as vertical straight lines. Only primary colors (red,
blue and yellow) and non-colors (black, white and grey) were allowed to use.

Another founding member of \textit{De Stijl}, Georges Vantongerloo, used methods
inspired from mathematics for his compsitions. One obvious example of this is
his painting "Composition Derived from the Equation y = -ax2 + bx + 18".

Mondrian on the other hand is known to only use intuition when creating his
works, moving composiotional elements for weeks, seeking a balanced composition,
until he was satisfied with the result. % Find Citation, better

% Still people were interested in this



% https://arthistoryproject.com/artists/piet-mondrian/the-collected-writings-of-piet-mondrian/general-principals-of-neo-plasticism/

% Livio2002: Mondrian mocking arithmetic computations with regard to his work, Intuation not Calculation
% Any kind of structure find in paintings are intuitive or non-concious apart from the basic axioms

% Mondrian axiom (cite Mondrian himself)

% - Mondrian obsessd with positiotioning. he spend long periods of time shifting a single line back and forth within a couple of millimetres, believing that a precise positioning was essential for capturing an aesthetic order that was “free of tension” [Deicher, S., (1995), Mondrian, Taschen (Koln), 1995]. Australian

% Historical background on Mondrian
% Motivation

\subsection{Related Works}

% TODO: Write more about Hill, maybe, see Bear for more material about this

In 1968 Hill \cite{Hill1968} analysed the network topology of Mondrian's paintings.
While he did not see value in metrical analysis of Mondrian paintings, he called
for a more detailed analysis of his paintings. He criticised earlier work for
the lack of statistics.

\subsubsection{Golden ratio}
% TODO: In popular science. Claims of golden ratio, charles Bouleau the painters secret geometry
% - Bergamini

% TODO: paraphrase
% False claims about artists allegedly using the Golden Ratio continue to spring
% up almost like mushrooms after the rain. One of these claims deserves some
% special attention, since it is repeated endlessly.
\cite{Livio2002}
% However mathematician Markowsky \cite{Markowsky1992} says merely superimposing

While it is said that many people have tried and failed analysing Mondrian's
paintings for certain ratios, there hasn't been many people actually testing
the hypotheses.

In a presentation from 2017 Tanaka and Miyanaga \cite{Tanaka2017} examine the use
of rectangle ratios, specifically the golden and silver ratio, in 10 late
Mondrian paintings. Their suggesting a preference of Mondrian for the silver
ratio.

However to the authors knowledge there is no publication discussing the question of
the use of certain ratios with a bigger data set.

\subsubsection{Computer-generated Mondrian}

% Noll “Computer Composition With Lines” 1964

% TODO: Write more about Fejs, read that paper again

Feijs (2004)\cite{Feijs2004} presents different techniques for generating images
resembling non-figurative Mondrian paintings from different times. He  concludes
that different kinds of algorithms for generating images can be used to
formalise the distinction between different types of Mondrian paintings.

% TODO: Better Skrodzki section

Skrodzki and Polthier (2018) \cite{Skrodzki2018} use computer models to generate
Mondrian-inspired three-dimensional pieces. They note however that while their
results are similar to Mondrian paintings, they do not all resemble Mondrian
paintings in a way that proportions are randomly chosen and not carefully.
