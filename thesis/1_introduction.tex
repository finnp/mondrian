% !TeX encoding = UTF-8
\section{Introduction}

% Introduce general problem area?

\subsection{Piet Mondrian}

The oeuvre of Piet Mondrian (1872-1944) consists of figurative paintings as well
as abstract compositions. Up until 1910 his works only depicted naturalistic
scenes like churches, trees, windmills or landscapes. From 1911 onwards his
paintings remain representational but are increasingly painted in a more
abstract way.

Around 1914 his compositions begin to be purely abstract and in 1917 he cofounds
the artistic movement and group \textit{De Stijl} also known as
\textit{Neoplasticism}. One of the primary objectives of the group is to reform
art by "abolishing natural form" \cite{wiki:manifest} altogether.

At some point around 1920 Mondrian further restricted his compositional elements
to rectangles and straight black lines. The lines were only allowed to run
horizontally or vertically. Only primary colors (red, blue and yellow) and
non-colors (black, white and grey) were allowed to be used.

Another founding member of \textit{De Stijl}, Georges Vantongerloo, used methods
inspired from mathematics for his compositions. One example of this is his
painting "Composition Derived from the Equation y = -ax2 + bx + 18".

Mondrian on the other hand is known to only use intuition when creating his
works, moving compositional elements for weeks, seeking a balanced composition,
until he was satisfied with the result.
% Find Citation, better
% Deicher, S., (1995), Mondrian, Taschen (Koln), 1995 - Mondrian obsessed with positioning. he spends long periods of time shifting a single line back and forth within a couple of millimetres, believing that a precise positioning was essential for capturing an aesthetic order that was “free of tension”. Australian

% TODO: Avoid Symmetry

Still many people have been interested in the question if Mondrian's
non-figurative paintings utilize certain compositional rules or techniques that
the painter might have used unconsciously. One approach that only became
feasible in the 1960s is to recreate art that resembles Mondrian's composition
using computers.

\subsection{Related Works}

There have been multiple attempts at creating art that resembles Mondrian's
compositions using computers. Feijs (2004)\cite{Feijs2004} presents different
techniques for generating images resembling non-figurative Mondrian paintings
from different periods. He concludes that different kinds of algorithms for
generating images can be used to formalise the distinction between different
types of Mondrian paintings.

Skrodzki and Polthier (2018) \cite{Skrodzki2018} use computer models to generate
Mondrian-inspired three-dimensional pieces using the KdTree data structures.
They note however that while their results are similar to Mondrian paintings,
they do not always resemble Mondrian paintings since proportions are randomly
chosen and not carefully.

There have been attempts to study the organisation and proportions of the
compositional elements used by Mondrian. In 1968 Hill \cite{Hill1968} analysed
the network topology of Mondrian's paintings. One of his findings shows the
avoidance of symmetry on a structural level. While he did not see value in
metrical analysis of Mondrian paintings, he called for a more detailed analysis
of the paintings. He also criticised earlier work for the lack of statistics and
inaccuracy of measurement.

Some sources \cite{bouleau1963,bergamini1980} suggest that Mondrian used golden
rectangles in his paintings. A golden rectangle is rectangle, where the ratio
between the sides length is the golden ratio $\phi \approx 1.618$. Other authors
however \cite{Livio2002,Markowsky1992} refuse those claims, finding fault with
the lack of evidence, which consists mostly just of exemplary superimposing
golden rectangles on paintings.

Despite this disagreement there have been very little attempts on actually
evaluating this question statistically. In a poster from 2017, Tanaka and
Miyanaga \cite{Tanaka2017} examine the use of rectangle ratios, specifically the
golden and silver ratio, in 10 late Mondrian paintings. They conclude that
Mondrian actually had a preference for the silver ratio $\delta_S = \approx 2.414$ instead.

However to the authors knowledge there is no publication examining the question of
the use of certain special ratios on Mondrian's paintings with a larger data
set.

% TODO: Could also be in discussion?
% More motivational, could be easier done with our program
% http://faculty.philosophy.umd.edu/jhbrown/mondriansbalance/index.html#19
% "The lesson is that we must exploit digital technology to the full and set Mondrian's
% designs into the relevant space of possibilities if we are ever to put judgments
% of aesthetic balance on a firm footing – which at present they are clearly not."
% "Such findings also suggest that we can put the formalist sector of aesthetic
% response on a firm footing only by much more extensive and rigorous
% experimentation with transforms of existing designs"

\subsection{Scope of the Thesis}

Providing numeric data to close the lack of foundation for analysis on the
neo-plastic compositions by Mondrian is the main objective of this thesis. The
goal is to create a program that uses methods of Computer Vision to extract the
compositional elements from images of the paintings. The program is restricted
to paintings from 1920-1937 that also always use black lines to separate the
rectangles and are not drawn on a diamond canvas.
