% !TeX encoding = UTF-8
\section{Introduction}


% Livio2002: Mondrian mocking arithmetic computations with regard to his work, Intuation not Calculation
% Any kind of structure find in paintings are intuitive or non-concious apart from the basic axioms

% Mondrian axiom (cite Mondrian himself)

% - Mondrian obsessd with positiotioning. he spend long periods of time shifting a single line back and forth within a couple of millimetres, believing that a precise positioning was essential for capturing an aesthetic order that was “free of tension” [Deicher, S., (1995), Mondrian, Taschen (Koln), 1995]. Australian

% Historical background on Mondrian
% Motivation

\subsection{Related Works}

% TODO: Write more about Hill, maybe, see Bear for more material about this

In 1968 Hill \cite{Hill1968} analysed the network topology of Mondrian's paintings.
While he did not see value in metrical analysis of Mondrian paintings, he called
for a more detailed analysis of his paintings. He criticised earlier work for
the lack of statistics.

\subsubsection{Golden ratio}
In a presentation from 2017 Tanaka and
Miyanaga\cite{Tanaka2017} examine the use of rectangle ratios, specifically the
golden and silver ratio, in 10 late Mondrian paintings. Their suggesting a
preference of Mondrian for the silver ratio.

% In popular science. Claims of golden ratio, charles Bouleau the painters secret geometry
% - Bergamini

% False claims about artists allegedly using the Golden Ratio continue to spring
% up almost like mushrooms after the rain. One of these claims deserves some
% special attention, since it is repeated endlessly.
\cite{Livio2002}
% However mathematician Markowsky \cite{Markowsky1992} says merely superimposing

% No publication with a substantial data basis.

\subsubsection{Computer-generated Mondrian}

% Noll “Computer Composition With Lines” 1964

% TODO: Write more about Fejs, read that paper again

Feijs (2004)\cite{Feijs2004} presents different techniques for generating images
resembling non-figurative Mondrian paintings from different times. He  concludes
that different kinds of algorithms for generating images can be used to
formalise the distinction between different types of Mondrian paintings.

% TODO: Better Skrodzki section

Skrodzki and Polthier (2018) \cite{Skrodzki2018} use computer models to generate
Mondrian-inspired three-dimensional pieces. They note however that while their
results are similar to Mondrian paintings, they do not all resemble Mondrian
paintings in a way that proportions are randomly chosen and not carefully.
