% !TeX encoding = UTF-8
\section{Discussion} \label{conclusion}

% Summary of the findings and conclusions


% What deserves further research?

The data we obtained did not show a preferred use of the golden or silver ratio
in the rectangles of Mondrian's paintings.




% Color distribution numbers close to Paretos Principle. Maybe 80% of the
% effect of the painting stem from the 20% of the colors. Psychological research?
% eye-tracking studies

\subsection{Red}

In "Art and Visual Perception" Rudolg Arnheim \cite{Arnheim1965} notes that "an
object in the upper part of the composition is heavier tha none in the lower;
and location at the right side makes more weight than location on the left."
Winner (1987) \cite{Winner1987} psychologically tested these principles by
showing abstract images and their horizontally or vertically flipped
counterparts. Their findings support the up-down-principles, but not the
left-right-principle.

Arnheim also notes that the color red is heavier than the color blue. A study
from \cite{Locher2005} found some support for this claim when showing
design-trained participants modified Mondrian paintings were some colors were
swopped.

It would be interesting to explore if the found preference by Mondrian to use
red in the top-left corner could be explained by a combination of these proposed
principles of balance or whether it is merely coincidental or habitual.


% Do digital idealized representation of Mondrian paintings are comparable
% to the actual paintings in terms of certain factors like perceptional balance

% TODO: Big is heaver than small (i think)

% More generally: More paintings by mondrian, other artists of De Stijl and
% artists in general
