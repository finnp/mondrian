% !TeX encoding = UTF-8
\section{Discussion} \label{conclusion}

The developed program is able with some certainty to detect rectangles in
a class of abstract Mondrian paintings. Human control and intervention is still
necessary to make sure the result is correct. By adjusting the input parameter
or in some cases manually editing the input images, all images were detected
successfully. 75\% of the images could be detected without manual intervention.

A first look in the resulting data showed some interesting finding that deserve
further investigation. The dataset that we obtained from the images is made
freely available for other applications, from statistical analysis to machine
learning.

The data does not suggest a preferred use of the golden or silver ratio in
rectangles. There only seems to be a general tendency towards squares.
Justifying continued research into this direction should depend on findings in
the psychology of aesthetics. So far the studies about the visually pleasing
nature of golden rectangles appear to contradict each other. Perceptional
studies about the silver ratios are missing entirely. However researchin  ratios
of other components, like the positioning of lines or the size of rectangles to
each other might lead to intersting results.

\subsection{Red}

In "Art and Visual Perception" Rudolg Arnheim \cite{Arnheim1965} notes that "an
object in the upper part of the composition is heavier tha none in the lower;
and location at the right side makes more weight than location on the left."
Winner (1987) \cite{Winner1987} psychologically tested these principles by
showing abstract images and their horizontally or vertically flipped
counterparts. Their findings support the up-down-principles, but not the
left-right-principle.

Arnheim also notes that the color red is heavier than the color blue. A study
from \cite{Locher2005} found some support for this claim when showing
design-trained participants modified Mondrian paintings were some colors were
swopped.

It would be interesting to explore if the found preference by Mondrian to use
red in the top-left corner could be explained by a combination of these proposed
principles of balance or whether it is merely coincidental or habitual.


% Do digital idealized representation of Mondrian paintings are comparable
% to the actual paintings in terms of certain factors like perceptional balance

% TODO: Big is heaver than small (i think)

% More generally: More paintings by mondrian, other artists of De Stijl and
% artists in general

% Color distribution numbers close to Paretos Principle. Maybe 80% of the
% effect of the painting stem from the 20% of the colors. Psychological research?
% eye-tracking studies
